\documentclass[10pt,t]{beamer}

% Paquets LaTeX %%%%%%%%%%%%%%%%%%%%%%%%%%%%%%%%%%%%%%%%%%%%%%%%%%%%%%%%%%%%%%

%% Une gestion correcte du français (en entrée et en sortie)
\usepackage[french]{babel}
\usepackage{type1ec}         % devant fontenc (cf type1ec.sty)
\usepackage[T1]{fontenc}     % devant inputenc (utf8 choisi en fonction de ça)
\usepackage[utf8]{inputenc}
\DeclareUnicodeCharacter{20AC}{\euro} % pour la saisie du caractère euro

%% Des "jolies" polices de caractères
\usepackage{lmodern}            % pour sf et tt
\usepackage{fourier}            % pour rm
\usepackage{bbm}                % pour les mathbbm

%% Plein de symboles
\usepackage{amssymb}            % Les symboles mathématiques de l'AMS
\usepackage{latexsym}           % Quelques symboles manquants dans LaTeX 2e
\usepackage{marvosym}           % Quelques symboles en vrac par Martin Vogel
\usepackage{wasysym}            % Quelques symboles en vrac par Roland Waldi
\usepackage{pifont}             % Les symboles Dingbats
\usepackage{textcomp}           % \textcopyleft
\usepackage[copyright]{ccicons} % Les (c) comme dans Creative Commons
\usepackage[official,right]{eurosym} % L'euro

%% Quelques paquets utiles
\usepackage{array}              % pour faciliter les styles de tableaux
\usepackage{relsize}            % pour faciliter le changement de taille des polices
\usepackage[normalem]{ulem}     % pour avoir des soulignements funky
\usepackage{tikz}               % pour les dessins portables
\usepackage{pgfpages}           % pour les présentations en double-écran

% Paramétrages Beamer %%%%%%%%%%%%%%%%%%%%%%%%%%%%%%%%%%%%%%%%%%%%%%%%%%%%%%%%
\usetheme{lyon2}
%\setbeameroption{show notes on second screen} % les notes sur le second écran

% Méta-données du document %%%%%%%%%%%%%%%%%%%%%%%%%%%%%%%%%%%%%%%%%%%%%%%%%%%

\title{Un modèle de cours à Lyon 2}
\subtitle{Avec \LaTeX{} et Beamer}
\author{Prénom NOM}
\institute{Université Lyon 2 \and CNRS}
\date{2015/2016}

%%%%%%%%%%%%%%%%%%%%%%%%%%%%%%%%%%%%%%%%%%%%%%%%%%%%%%%%%%%%%%%%%%%%%%%%%%%%%%

\begin{document}

\begin{frame}[plain,label=titre]
  \titlepage
\end{frame}

\begin{frame}
  \frametitle{Plan}
  \tableofcontents
\end{frame}

%%%%%%%%%%%%%%%%%%%%%%%%%%%%%%%%%%%%%%%%%%%%%%%%%%%%%%%%%%%%%%%%%%%%%%%%%%%%%%

\begin{frame}[fragile]% [shrink] si on veut forcer à tenir sur une seul frame
  \frametitle{Introduction}

  \structure{Un exemple}

  \begin{itemize}
  \item Cette présentation est produite par le fichier \verb|modele-cours-lyon.tex|
  \item Elle est conçue pour des cours magistraux
  \item Son contenu est un exemple pour comprendre comment

    \begin{itemize}
    \item composer avec \LaTeX{}
    \item préparer une présentation avec Beamer
    \end{itemize}
  \end{itemize}

  \pause

  \structure{Un modèle}

  \begin{itemize}
  \item Elle constitue une base modèle pour faire d'autres cours
  \item Pour faire un nouveau cours il faut 
    
    \begin{enumerate}
    \item copier les fichiers 

      \begin{itemize}
      \item \verb|beamerthemelyon2.sty| qui est le thème Beamer
      \item \verb|logo-univ-lyon2.pdf| qui est le logo de Lyon 2 utilisé
      \item \verb|logo-gate.pdf| qui est le logo du GATE utilisé
      \end{itemize}

    \item copier, renommer puis modifier le fichier \verb|modele-cours-lyon2.tex|
    \item compiler le fichier
    \end{enumerate}
  \end{itemize}

  \pause

  \structure{La seconde partie est une courte référence \LaTeX{}}
\end{frame}

%%%%%%%%%%%%%%%%%%%%%%%%%%%%%%%%%%%%%%%%%%%%%%%%%%%%%%%%%%%%%%%%%%%%%%%%%%%%%%

\section{Faire une présentation en \LaTeX{} avec Beamer}

\subsection{Les bases}
\input{01-bases}

\subsection{Des conseils de productions}
\input{02-conseils}

\section{Un résumé mode d'emploi}

\subsection{\LaTeX{}}
\input{03-latex}

\subsection{Beamer}
\input{04-beamer}

\againframe{titre} % Replace une diapo nommée avec l'option label

%%%%%%%%%%%%%%%%%%%%%%%%%%%%%%%%%%%%%%%%%%%%%%%%%%%%%%%%%%%%%%%%%%%%%%%%%%%%%%

\end{document}
