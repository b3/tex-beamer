\documentclass[10pt,t]{beamer}
% [aspectratio=169] pour du 16/9

\usepackage[french]{babel}
\usepackage{type1ec}
\usepackage[T1]{fontenc}
\usepackage[utf8]{inputenc}
\usepackage{graphicx}

% include theme-options.tex

% include simple-metadata.tex

\backgroundtitle{tex-friendly-zone}

%%%%%%%%%%%%%%%%%%%%%%%%%%%%%%%%%%%%%%%%%%%%%%%%%%%%%%%%%%%%%%%%%%%%%%%%%%%%%%

\begin{document}

\maketitle

%%%%%%%%%%%%%%%%%%%%%%%%%%%%%%%%%%%%%%%%%%%%%%%%%%%%%%%%%%%%%%%%%%%%%%%%%%%%%%

\section{Introduction}

% [shrink] si on veut forcer à tenir sur une seul frame
% [plain] si on ne veut pas de décoration
% [fragile] s'il y du texte verbatim
\begin{frame}{Introduction} 
  \structure{Un exemple}

  \begin{itemize}
    % include simple-exemple.tex
  \item Son contenu est un exemple pour comprendre comment

    \begin{itemize}
    \item composer avec \LaTeX{}
    \item préparer une présentation avec Beamer
    \end{itemize}

  \end{itemize}

  \pause

  \structure{Un modèle}

  % include simple-modele.tex
\end{frame}

\end{document}
