\documentclass[10pt,t]{beamer}
% [aspectratio=169] pour du 16/9

\usepackage[french]{babel}
\usepackage{type1ec}
\usepackage[T1]{fontenc}
\usepackage[utf8]{inputenc}
\usepackage{graphicx}

\usetheme{sif}

\title[SIF]{Un modèle simple}
\subtitle{Avec \LaTeX{} et Beamer}
\author{Bruno BEAUFILS}
\institute[]{GT Moyens Informatiques}
\date{21 janvier 2021}

%%%%%%%%%%%%%%%%%%%%%%%%%%%%%%%%%%%%%%%%%%%%%%%%%%%%%%%%%%%%%%%%%%%%%%%%%%%%%%

\begin{document}

\maketitle

%%%%%%%%%%%%%%%%%%%%%%%%%%%%%%%%%%%%%%%%%%%%%%%%%%%%%%%%%%%%%%%%%%%%%%%%%%%%%%

% [shrink] si on veut forcer à tenir sur une seul frame
% [plain] si on ne veut pas de décoration
% [fragile] s'il y du texte verbatim
\begin{frame}{Introduction} 
  \structure{Un exemple}

  \begin{itemize}
  \item Cette présentation est produite par \texttt{\textbf{exemple-sif-simple.tex}}
  \item Elle est conçue pour la \href{https://www.societe-informatique-de-france.fr}{Société Informatique de France}
  \item Son contenu est un exemple pour comprendre comment

    \begin{itemize}
    \item composer avec \LaTeX{}
    \item préparer une présentation avec Beamer
    \end{itemize}

  \end{itemize}

  \pause

  \structure{Un modèle}

  \begin{itemize}
  \item Elle constitue un modèle pour faire d'autres présentations :
      
    \begin{enumerate}
    \item créer un nouveau dossier ;
    \item y copier les fichiers 
      
      \begin{itemize}
      \item \texttt{\textbf{beamerthemesif.sty}}, qui est le thème Beamer ;
      \item \texttt{\textbf{logo-sif.pdf}}, qui est le logo de la SIF ;
      \item \texttt{\textbf{bandeau-sif.pdf}}, qui est le bandeau de bas de page ;
      \end{itemize}
      
    \item y copier, renommer puis modifier le fichier \texttt{\textbf{exemple-sif-simple.tex}} ;
    \item y compiler le fichier.
    \end{enumerate}
  \end{itemize}
\end{frame}

\end{document}
