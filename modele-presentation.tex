\documentclass[10pt,t]{beamer}

% Paquets LaTeX %%%%%%%%%%%%%%%%%%%%%%%%%%%%%%%%%%%%%%%%%%%%%%%%%%%%%%%%%%%%%%

%% Une gestion correcte du français (en entrée et en sortie)
\usepackage[french]{babel}
\usepackage{type1ec}         % devant fontenc (cf type1ec.sty)
\usepackage[T1]{fontenc}     % devant inputenc (utf8 choisi en fonction de ça)
\usepackage[utf8]{inputenc}
\DeclareUnicodeCharacter{20AC}{\euro} % pour la saisie du caractère euro

%% Plein de symboles
\usepackage{amssymb}            % Les symboles mathématiques de l'AMS
\usepackage{latexsym}           % Quelques symboles manquants dans LaTeX 2e
\usepackage{marvosym}           % Quelques symboles en vrac par Martin Vogel
\usepackage{wasysym}            % Quelques symboles en vrac par Roland Waldi
\usepackage{pifont}             % Les symboles Dingbats
\usepackage{textcomp}           % \textcopyleft
\usepackage[copyright]{ccicons} % Les (c) comme dans Creative Commons
\usepackage[official,right]{eurosym} % L'euro

%% Quelques paquets utiles
\usepackage{array}              % pour faciliter les styles de tableaux
\usepackage{relsize}            % pour faciliter le changement de taille des polices
\usepackage[normalem]{ulem}     % pour avoir des soulignements funky
\usepackage{tikz}               % pour les dessins portables
\usepackage{pgfpages}           % pour les présentations en double-écran

% Paramétrages Beamer %%%%%%%%%%%%%%%%%%%%%%%%%%%%%%%%%%%%%%%%%%%%%%%%%%%%%%%%
% include theme.tex
%\setbeameroption{show notes on second screen} % les notes sur le second écran

% Méta-données du document %%%%%%%%%%%%%%%%%%%%%%%%%%%%%%%%%%%%%%%%%%%%%%%%%%%

% include presentation-metadata.tex

%%%%%%%%%%%%%%%%%%%%%%%%%%%%%%%%%%%%%%%%%%%%%%%%%%%%%%%%%%%%%%%%%%%%%%%%%%%%%%

\begin{document}

\begin{frame}[plain]
  \titlepage
\end{frame}

\begin{frame}
  \frametitle{Plan}
  \tableofcontents % il faut compiler deux fois pour mettre à jour la TDM
\end{frame}

%%%%%%%%%%%%%%%%%%%%%%%%%%%%%%%%%%%%%%%%%%%%%%%%%%%%%%%%%%%%%%%%%%%%%%%%%%%%%%

\section{Introduction}

%%%%%%%%%%%%%%%%%%%%%%%%%%%%%%%%%%%%%%%%%%%%%%%%%%%%%%%%%%%%%%%%%%%%%%%%%%%%%%

\begin{frame}% [shrink] si on veut forcer à tenir sur une seul frame (c'est mal)
  \frametitle{Introduction}

  \structure{Un exemple}

  \begin{itemize}
    % include presentation-exemple.tex
  \end{itemize}

  \pause

  \structure{Un modèle}

  % include presentation-modele.tex
  
\end{frame}

%%%%%%%%%%%%%%%%%%%%%%%%%%%%%%%%%%%%%%%%%%%%%%%%%%%%%%%%%%%%%%%%%%%%%%%%%%%%%%

\section{Le reste de la présentation}

%%%%%%%%%%%%%%%%%%%%%%%%%%%%%%%%%%%%%%%%%%%%%%%%%%%%%%%%%%%%%%%%%%%%%%%%%%%%%%

\begin{frame}
  \frametitle{La première diapo}
  
  \structure{Un truc structurant}
  
  \begin{itemize}
  \item un point en \emph{emphase}
    \note{Ici il faut dire un bidule sur l'emphase}
  \item un autre point \underline{souligné}
  \item un point très \textbf{important}
  \end{itemize} 
  
  \pause  
  
  \structure{Un autre truc qui structure}
  
  \begin{itemize}
    % include presentation-liens.tex
  \item un autre point avec de l'€
  \end{itemize} 
\end{frame}

%%%%%%%%%%%%%%%%%%%%%%%%%%%%%%%%%%%%%%%%%%%%%%%%%%%%%%%%%%%%%%%%%%%%%%%%%%%%%%

\begin{frame}
  \frametitle{La seconde diapo}
  
  \structure{Un truc}
  
  \begin{block}{Un bloc important}
    Les points suivants définissent l'importance
    \begin{itemize}
    \item un truc
    \item un bidule
    \end{itemize}
  \end{block}
  
  \pause  
  
  \structure{Un truc}
  
  \begin{itemize}
  \item un point
  \item un truc mise en \alert{avant} avec de l'€
  \end{itemize} 
  \note{il faut répondre aux questions}
\end{frame}

%%%%%%%%%%%%%%%%%%%%%%%%%%%%%%%%%%%%%%%%%%%%%%%%%%%%%%%%%%%%%%%%%%%%%%%%%%%%%% 

\end{document}
