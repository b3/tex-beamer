%%%%%%%%%%%%%%%%%%%%%%%%%%%%%%%%%%%%%%%%%%%%%%%%%%%%%%%%%%%%%%%%%%%%%%%%%%%%%%

\begin{frame}[fragile]
  \frametitle{Conseils d'édition}

  Avoir des source \LaTeX{} propres permet 

  \begin{enumerate}
  \item de se repérer rapidement dans la présentation
  \item de détecter plus facilement les erreurs de compilation
  \end{enumerate}

  \alert{Il faut s'astreindre à des conventions d'écriture des sources}

  \pause

  \begin{itemize}
  \item Toujours décaler le contenu d'un environnement vers la droite

    \begin{itemize}
    \item Objectif : faire apparaître la structure
    \item Même largeur à chaque fois (2 espaces par exemple)
    \item Additioner les décalages à chaque descente en profondeur      
    \end{itemize}

  \item Spécifité des environnements de liste

    \begin{itemize}
    \item Les items de liste doivent être alignées avec le début et la fin de l'environnement
    \item Laisser une ligne vide devant le début d'une liste
    \item Laisser une ligne vide après la fin d'une liste
    \end{itemize}

  \item Éviter la multiplication des lignes vides qui ne servent à rien
  \item Entourer les diapos avec des longues lignes de commentaires (\%)
  \end{itemize}
\end{frame}

%%%%%%%%%%%%%%%%%%%%%%%%%%%%%%%%%%%%%%%%%%%%%%%%%%%%%%%%%%%%%%%%%%%%%%%%%%%%%%
