%%%%%%%%%%%%%%%%%%%%%%%%%%%%%%%%%%%%%%%%%%%%%%%%%%%%%%%%%%%%%%%%%%%%%%%%%%%%%%

\begin{frame}
  \frametitle{Courte référence \LaTeX{}}
  \framesubtitle{Listes, emphases, liens}

  \structure{Une liste standard}
  
  \begin{itemize}
  \item un point en \emph{emphase}
    \note{Ici il faut dire un bidule sur l'emphase}
  \item un point \textbf{important}
  \item un autre point \underline{souligné}
  \end{itemize} 
  
  \pause  
  
  \structure{Une liste numérotée}
  
  \begin{enumerate}
  \item un point avec un lien vers le \href{http://www.univ-lille1.fr}{site web de l'Université}
  \item un autre point sur le site web : \url{http://www.univ-lille1.fr}
  \item un choix avec de l'€
  \end{enumerate}

  \pause

  \structure{Une liste de description}

  Quelques environnements permettent des structurations 

  \begin{description}
  \item[itemize] définit une liste standard
  \item[enumerate] définit une liste ordonnée
  \item[tabular] permet de faire des tableaux
  \item[displaymath] permet d'afficher des mathématiques
  \item[equation] permet d'afficher une équation et de la numéroter
  \end{description}
\end{frame}

%%%%%%%%%%%%%%%%%%%%%%%%%%%%%%%%%%%%%%%%%%%%%%%%%%%%%%%%%%%%%%%%%%%%%%%%%%%%%%

\begin{frame}[allowframebreaks] % une grande diapo qui sera découpée auto (pas de pause dedans)
  \frametitle{Courte référence \LaTeX{}}
  \framesubtitle{Mathématiques}

  \structure{3 possibilités de saisir du contenu \emph{mathématiques}}

  \begin{enumerate}
  \item En ligne : $ \sum_{i=0}^{i=10} a_{i} $

  \item En affiche :
    \begin{displaymath}
      \sum_{i=0}^{i=10} a_{i}
    \end{displaymath}

  \item En équation pour y faire référence plus tard
    \begin{equation}
      \label{eq:1}
      F(x) = \int_{a}^{b}\sin(x)dx 
    \end{equation}
  \end{enumerate}
  
  \structure{Des fonctionnalités accessibles uniquement en mode mathématique}

  \begin{itemize}
  \item Les lettres grecques et exotiques sont accessibles facilement :

    \begin{itemize}
    \item en minuscules : $ \alpha, \beta, \gamma, \delta, \epsilon, \zeta, \eta, \theta, \iota, \kappa, \lambda, \mu, \nu, \xi, \pi, \rho, \sigma, \tau, \upsilon, \phi, \chi, \psi, \omega $
    \item en capitales : $ \Gamma, \Delta, \Theta, \Lambda, \Xi, \Pi, \Sigma, \Upsilon, \Phi, \Psi, \Omega $
    \end{itemize}

  \item Les notations d'ensemble sont composables : $ \mathbbm{N}, \mathbbm{Z}^{+}, \mathbbm{Q}, \mathbbm{R} $

  \item Beaucoup de symboles mathématiques et autres sont disponibles

    \begin{itemize}
    \item $ \neq, \approx, \simeq, \le, \ge, \iff, \equiv, \in, \infty, \forall, \exists, \subset, \subseteq, \times, \emptyset, \rightarrow, \qed, \prod, \coprod, \nabla, \partial, \prec, \preceq, \cdots $
    \item $ \spadesuit, \heartsuit, \diamondsuit, \clubsuit, \ell, \sharp, \flat, \cdots $
    \item \url{http://detexify.kirelabs.org} permet de les trouver en les dessinant
    \end{itemize}

  \item On a plein de constructions

    \begin{description}
    \item[racine] $ \sqrt[3]{x + y} $
    \item[fraction] $ \frac{x\times 3}{4} $
    \item[relation] $ x \stackrel{i+j}{\longmapsto} y $
    \item[somme] $ \sum_{0}^{\infty} x_{i} $
    \item[produit] $ \prod_{0}^{\infty} x_{i} $
    \item[intégral] $ \int_{a}^{b} \pi_{i} $
    \item[regroupement]
      $ \underline{dessous} $,
      $ \overbrace{dessus}^{i\rightarrow\infty} $,
      sur le côté
      \begin{math} % pareil que $
        f(x) = \left\{
          \begin{array}{l@{~:~}r}
            x = 0 & x \\
            x \neq 0 & \frac{x}{2}
          \end{array}
        \right.
      \end{math}
    \end{description}

  \item On a des points de suite dans tous les sens $ \dots, \cdots, \vdots, \ddots $
  \end{itemize}
\end{frame}

%%%%%%%%%%%%%%%%%%%%%%%%%%%%%%%%%%%%%%%%%%%%%%%%%%%%%%%%%%%%%%%%%%%%%%%%%%%%%%

\begin{frame}[fragile]
  \frametitle{Courte référence \LaTeX{}}
  \framesubtitle{Avancé}

  \structure{Tableaux}

  \begin{itemize}
  \item Une suite de lignes arrangées en colonnes via \verb|tabular| ou \verb|array| (maths)
  \item Il doit y avoir une ligne blanche devant et derrière
    
    \begin{tabular}{|c|r|l|}
      \firsthline
      colonnes & définition         & paramètre de l'environnement via l, c, r  \\ \cline{2-3}
               & traits verticaux   & \verb!|! dans la définition des colonnes  \\ \hline
      lignes   & terminaison        & $\backslash\backslash$ en fin de ligne    \\ \cline{2-3}
               & traits horizontaux & sur toute la largeur via \verb|\hline|    \\ \cline{2-3}
               &                    & sous certaines colonnes via \verb|\cline| \\ \hline
    \end{tabular}
  \end{itemize}

  \structure{Références aux équations}

  \begin{itemize}
  \item on la nomme avec \verb|\label{}|
  \item on y fait référence avec \verb|\ref{}|
  \item par exemple l'équation \ref{eq:1} utilise une intégrale
  \end{itemize}
\end{frame}

%%%%%%%%%%%%%%%%%%%%%%%%%%%%%%%%%%%%%%%%%%%%%%%%%%%%%%%%%%%%%%%%%%%%%%%%%%%%%%

\begin{frame}[fragile]
  \frametitle{Courte référence \LaTeX{}}
  \framesubtitle{Avancé}

  \structure{Caractères spéciaux}

  \begin{center}
    \begin{tabular}{rl}
      \emph{caractère} & \emph{commande} \\ \hline
      \# & \verb|\#| \\
      \$ & \verb|\$| \\
      \% & \verb|\%| \\
      \& & \verb|\&| \\
      \textasciitilde & \verb|\textasciitilde| \\
      \_ & \verb|\_| \\
      \textasciicircum & \verb|\textasciicircum| \\
      $\backslash$ & \verb|$\backslash$| \\
      \{ & \verb|\{| \\
      \} & \verb|\}| \\
    \end{tabular}
  \end{center}
\end{frame}

%%%%%%%%%%%%%%%%%%%%%%%%%%%%%%%%%%%%%%%%%%%%%%%%%%%%%%%%%%%%%%%%%%%%%%%%%%%%%%
