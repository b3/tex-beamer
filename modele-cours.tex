\documentclass[10pt,t]{beamer}

% Paquets LaTeX %%%%%%%%%%%%%%%%%%%%%%%%%%%%%%%%%%%%%%%%%%%%%%%%%%%%%%%%%%%%%%

%% Une gestion correcte du français (en entrée et en sortie)
\usepackage[french]{babel}
\usepackage{type1ec}         % devant fontenc (cf type1ec.sty)
\usepackage[T1]{fontenc}     % devant inputenc (utf8 choisi en fonction de ça)
\usepackage[utf8]{inputenc}
\DeclareUnicodeCharacter{20AC}{\euro} % pour la saisie du caractère euro

%% Plein de symboles
\usepackage{amssymb}            % Les symboles mathématiques de l'AMS
\usepackage{latexsym}           % Quelques symboles manquants dans LaTeX 2e
\usepackage{marvosym}           % Quelques symboles en vrac par Martin Vogel
\usepackage{wasysym}            % Quelques symboles en vrac par Roland Waldi
\usepackage{pifont}             % Les symboles Dingbats
\usepackage{textcomp}           % \textcopyleft
\usepackage[copyright]{ccicons} % Les (c) comme dans Creative Commons
\usepackage[official,right]{eurosym} % L'euro

%% Quelques paquets utiles
\usepackage{array}              % pour faciliter les styles de tableaux
\usepackage{relsize}            % pour faciliter le changement de taille des polices
\usepackage[normalem]{ulem}     % pour avoir des soulignements funky
\usepackage{tikz}               % pour les dessins portables
\usepackage{pgfpages}           % pour les présentations en double-écran

% Paramétrages Beamer %%%%%%%%%%%%%%%%%%%%%%%%%%%%%%%%%%%%%%%%%%%%%%%%%%%%%%%%
% include theme.tex
%\setbeameroption{show notes on second screen} % les notes sur le second écran

% Méta-données du document %%%%%%%%%%%%%%%%%%%%%%%%%%%%%%%%%%%%%%%%%%%%%%%%%%%

% include cours-metadata.tex

%%%%%%%%%%%%%%%%%%%%%%%%%%%%%%%%%%%%%%%%%%%%%%%%%%%%%%%%%%%%%%%%%%%%%%%%%%%%%%

\begin{document}

\begin{frame}[plain,label=titre]
  \titlepage
\end{frame}

\begin{frame}
  \frametitle{Plan}
  \tableofcontents % il faut compiler deux fois pour mettre à jour la TDM
\end{frame}

%%%%%%%%%%%%%%%%%%%%%%%%%%%%%%%%%%%%%%%%%%%%%%%%%%%%%%%%%%%%%%%%%%%%%%%%%%%%%%

\section*{Introduction}

%%%%%%%%%%%%%%%%%%%%%%%%%%%%%%%%%%%%%%%%%%%%%%%%%%%%%%%%%%%%%%%%%%%%%%%%%%%%%%

\begin{frame}% [shrink] si on veut forcer à tenir sur une seul frame
  \frametitle{Introduction}

  \structure{Un exemple}

  \begin{itemize}
    % include cours-exemple.tex
  \item Son contenu est un exemple pour comprendre comment

    \begin{itemize}
    \item composer avec \LaTeX{}
    \item préparer une présentation avec Beamer
    \end{itemize}

  \end{itemize}

  \pause

  \structure{Un modèle}

  % include cours-modele.tex

  \pause

  \structure{La seconde partie est une courte référence \LaTeX{}}
\end{frame}

%%%%%%%%%%%%%%%%%%%%%%%%%%%%%%%%%%%%%%%%%%%%%%%%%%%%%%%%%%%%%%%%%%%%%%%%%%%%%%

\section{Faire une présentation en \LaTeX{} avec Beamer}

%%%%%%%%%%%%%%%%%%%%%%%%%%%%%%%%%%%%%%%%%%%%%%%%%%%%%%%%%%%%%%%%%%%%%%%%%%%%%%

\subsection{Les bases}
\input{01-bases}

\subsection{Des conseils de productions}
\input{02-conseils}

%%%%%%%%%%%%%%%%%%%%%%%%%%%%%%%%%%%%%%%%%%%%%%%%%%%%%%%%%%%%%%%%%%%%%%%%%%%%%

\section{Une petite référence}

%%%%%%%%%%%%%%%%%%%%%%%%%%%%%%%%%%%%%%%%%%%%%%%%%%%%%%%%%%%%%%%%%%%%%%%%%%%%%%

\subsection{\LaTeX{}}
\input{03-latex}

\subsection{Beamer}
\input{04-beamer}

\subsection{Références}
\input{05-references}

\againframe{titre} % Replace une diapo nommée avec l'option label

%%%%%%%%%%%%%%%%%%%%%%%%%%%%%%%%%%%%%%%%%%%%%%%%%%%%%%%%%%%%%%%%%%%%%%%%%%%%%%

\end{document}
