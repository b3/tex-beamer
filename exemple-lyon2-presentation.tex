\documentclass[10pt,t]{beamer}
% [aspectratio=169] pour une présentation en 16/9

% Tout ce qui suit un % est considéré comme un commentaire par LaTeX

% Paquets LaTeX %%%%%%%%%%%%%%%%%%%%%%%%%%%%%%%%%%%%%%%%%%%%%%%%%%%%%%%%%%%%%%

%% Une gestion correcte du français (en entrée et en sortie)
\usepackage[french]{babel}
\usepackage{type1ec}         % devant fontenc (cf type1ec.sty)
\usepackage[T1]{fontenc}     % devant inputenc (utf8 choisi en fonction de ça)
\usepackage[utf8]{inputenc}
\DeclareUnicodeCharacter{20AC}{\euro} % pour la saisie du caractère euro

%% Plein de symboles
\usepackage{amssymb}            % Les symboles mathématiques de l'AMS
\usepackage{latexsym}           % Quelques symboles manquants dans LaTeX 2e
\usepackage{marvosym}           % Quelques symboles en vrac par Martin Vogel
\usepackage{wasysym}            % Quelques symboles en vrac par Roland Waldi
\usepackage{pifont}             % Les symboles Dingbats
\usepackage{textcomp}           % \textcopyleft
\usepackage[copyright]{ccicons} % Les (c) comme dans Creative Commons
\usepackage[official,right]{eurosym} % L'euro

%% Quelques paquets utiles
\usepackage{array}              % pour faciliter les styles de tableaux
\usepackage{relsize}            % pour faciliter le changement de taille des polices
\usepackage[normalem]{ulem}     % pour avoir des soulignements funky
\usepackage{tikz}               % pour les dessins portables
\usepackage{pgfpages}           % pour les présentations en double-écran

% Paramétrages Beamer %%%%%%%%%%%%%%%%%%%%%%%%%%%%%%%%%%%%%%%%%%%%%%%%%%%%%%%%
\usetheme{lyon2}
%\setbeameroption{show notes on second screen} % les notes sur le second écran
%\AtBeginPart[]{}       % enlever les diapos d'introduction des parties
%\AtBeginSection[]{}    % enlever les diapos d'introduction des sections
\AtBeginSubsection[]{} % enlever les diapos d'introduction des sous-sections

% Méta-données du document %%%%%%%%%%%%%%%%%%%%%%%%%%%%%%%%%%%%%%%%%%%%%%%%%%%

\title{Un modèle de présentation}
\subtitle{Avec \LaTeX{} et Beamer}
\author{Prénom NOM \and Autre-Prénom AUTRENOM}
\institute{Université Lyon~2}
\date{1\ier{} janvier 2016}

%%%%%%%%%%%%%%%%%%%%%%%%%%%%%%%%%%%%%%%%%%%%%%%%%%%%%%%%%%%%%%%%%%%%%%%%%%%%%%

\begin{document}

\maketitle

\begin{frame}
  \frametitle{Plan}
  \tableofcontents % il faut compiler deux fois pour mettre à jour la TDM
\end{frame}

%%%%%%%%%%%%%%%%%%%%%%%%%%%%%%%%%%%%%%%%%%%%%%%%%%%%%%%%%%%%%%%%%%%%%%%%%%%%%%

\section{Introduction}

%%%%%%%%%%%%%%%%%%%%%%%%%%%%%%%%%%%%%%%%%%%%%%%%%%%%%%%%%%%%%%%%%%%%%%%%%%%%%%

\begin{frame}% [shrink] si on veut forcer à tenir sur une seul frame (c'est mal)
  \frametitle{Introduction}

  \structure{Un exemple}

  \begin{itemize}
  \item Cette présentation est produite par \texttt{\textbf{exemple-lyon2-presentation.tex}}
  \item Elle est conçue pour des présentations recherche ou administrative des personnels de Lyon~2
  \end{itemize}

  \pause

  \structure{Un modèle}

  \begin{itemize}
  \item Elle constitue un modèle pour faire d'autres présentations :
    
    \begin{enumerate}
    \item créer un nouveau dossier ;
    \item y copier les fichiers 

      \begin{itemize}
      \item \texttt{\textbf{beamerthemelyon2.sty}}, qui est le thème Beamer,
      \item \texttt{\textbf{logo-univ-lyon2-2018.pdf}}, qui est le logo de Lyon~2 utilisé ;
      \end{itemize}

    \item y copier, renommer puis modifier le fichier \texttt{\textbf{exemple-lyon2-presentation.tex}} ;
    \item y compiler le fichier.
    \end{enumerate}
  \end{itemize}
  
\end{frame}

%%%%%%%%%%%%%%%%%%%%%%%%%%%%%%%%%%%%%%%%%%%%%%%%%%%%%%%%%%%%%%%%%%%%%%%%%%%%%%

\section{Le reste de la présentation}

%%%%%%%%%%%%%%%%%%%%%%%%%%%%%%%%%%%%%%%%%%%%%%%%%%%%%%%%%%%%%%%%%%%%%%%%%%%%%%

\begin{frame}
  \frametitle{La première diapo}
  
  \structure{Un truc structurant}
  
  \begin{itemize}
  \item un point en \emph{emphase}
    \note{Ici il faut dire un bidule sur l'emphase}
  \item un autre point \underline{souligné}
  \item un point très \textbf{important}
  \end{itemize} 
  
  \pause  
  
  \structure{Un autre truc qui structure}
  
  \begin{itemize}
  \item un point sur le \href{https://www.univ-lyon2.fr}{site web de Lyon~2}
  \item un autre point sur le site web : \url{https://www.univ-lyon2.fr}
  \item un autre point avec de l'€
  \end{itemize} 
\end{frame}

%%%%%%%%%%%%%%%%%%%%%%%%%%%%%%%%%%%%%%%%%%%%%%%%%%%%%%%%%%%%%%%%%%%%%%%%%%%%%%

\begin{frame}
  \frametitle{La seconde diapo}
  
  \structure{Un truc}
  
  \begin{block}{Un bloc important}
    Les points suivants définissent l'importance
    \begin{itemize}
    \item un truc
    \item un bidule
    \end{itemize}
  \end{block}
  
  \pause  
  
  \structure{Un truc}
  
  \begin{itemize}
  \item un point
  \item un truc mise en \alert{avant} avec de l'€
  \end{itemize} 
  \note{il faut répondre aux questions}
\end{frame}

%%%%%%%%%%%%%%%%%%%%%%%%%%%%%%%%%%%%%%%%%%%%%%%%%%%%%%%%%%%%%%%%%%%%%%%%%%%%%% 

\end{document}
